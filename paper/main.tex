\documentclass[a4paper, margin=1inch]{article}
\usepackage[margin=35mm]{geometry}

\synctex=1
\pdfoutput=1 

\usepackage[utf8]{inputenc}
\usepackage[T1]{fontenc}
\usepackage{lmodern}
\usepackage{microtype}

\usepackage[USenglish]{babel}

\usepackage{graphicx} % Required for inserting images
\usepackage{xcolor}
\usepackage{amsmath,amssymb,amsthm}
\usepackage[]{hyperref}
\hypersetup{
    colorlinks,
    linkcolor={red!50!black},
    citecolor={blue!50!black},
    urlcolor=.
}
\usepackage{float}
\usepackage[capitalize]{cleveref}
\usepackage{todonotes}
\usepackage{tabularx}
\usepackage[square,numbers]{natbib}
\usepackage{tikz}
\usetikzlibrary{calc}
\usetikzlibrary{positioning,backgrounds,patterns,calc,matrix,shapes,decorations.pathreplacing,decorations.pathmorphing,math}
\tikzset{crossing/.style={cross out, draw=red, minimum size=2*(#1-\pgflinewidth), inner sep=0pt, outer sep=1pt, very thick}, crossing/.default={4pt}}
\usetikzlibrary{arrows.meta,bending,decorations.markings,hobby,patterns,calc}
\newcolumntype{C}[1]{>{\centering\let\newline\\\arraybackslash\hspace{0pt}}m{#1}}

\usepackage{thmtools, thm-restate}
\newtheorem{redrule}{Reduction rule}
\newtheorem{claim}{Claim}
\newtheorem{lemma}{Lemma}
\newtheorem{corollary}{Corollary}
\newtheorem{observation}{Observation}
\newtheorem{proposition}{Proposition}
\newtheorem{definition}{Definition}
\newtheorem{theorem}{Theorem}

% Algorithms
\usepackage[ruled, vlined, linesnumbered, nofillcomment]{algorithm2e}
% taken from https://tex.stackexchange.com/questions/162207/algorithm2e-comment-style
\newcommand\commentstyle[1]{\footnotesize\ttfamily\textcolor{blue}{#1}}
\DontPrintSemicolon
\SetKwInOut{Input}{Input}
\SetKwInOut{Output}{Output}
\SetKwProg{Procedure}{Procedure}{}{}

%%%%%%%%%%%%%%%%%%%%%%%%%%%%%%%%%%%%%%%%%%%%%%%%%%%%%%
\newtheorem{assumption}{Assumption}
\newcommand{\cclass} [1] {\textnormal{\textsf{#1}}}

\usepackage{graphicx}
\usepackage{wrapfig}
\graphicspath{ {./figs/} }

\newcommand{\problemdef}[3]{
	\begin{center}
	\begin{minipage}{0.95\columnwidth}
		\noindent
		\textsc{#1}
		\vspace{5pt}\\
		\setlength{\tabcolsep}{3pt}
		\begin{tabularx}{\textwidth}{@{}lX@{}}
			\textbf{Input:}     & #2 \\
			\textbf{Question:}  & #3
		\end{tabularx}
	\end{minipage}
	\end{center}
}
\newcommand{\optproblemdef}[3]{
	\begin{center}
	\begin{minipage}{0.95\columnwidth}
		\noindent
		#1
		\vspace{5pt}\\
		\setlength{\tabcolsep}{3pt}
		\begin{tabularx}{\textwidth}{@{}lX@{}}
			\textbf{Input:}     & #2 \\
			\textbf{Task:}  & #3
		\end{tabularx}
	\end{minipage}
	\end{center}
}

\newcommand{\boxproblem}[4]{
    \begin{center}   
        \fbox{~\begin{minipage}{.97\textwidth}
            \vspace{2pt} 
            \noindent
            \normalsize\textsc{#1}
            \vspace{1pt}

            \setlength{\tabcolsep}{3pt}
            \renewcommand{\arraystretch}{1.0}
            \begin{tabularx}{\textwidth}{@{}lX@{}}
                \normalsize\textbf{Input:}       & \normalsize#2 \\
                \normalsize\textbf{Question:}    & \normalsize#3 \\
                \normalsize\textbf{Parameter: }  & \normalsize#4
            \end{tabularx}
        \end{minipage}}
    \end{center}
}
%%%%%%%%%%%%%%%%%%%%%%%%%%%%%%%%%%%%%%%%%%%%%%%%%%%%%%
\usepackage{authblk}
\newcommand{\email}[1]{ \href{mailto:#1}{\texttt{{#1}}}}

\newcommand{\MVVG}{\textsc{Maximum Value Vertex Guard}}
\newcommand{\MLVG}{\textsc{Maximum Length Vertex Guard}}
\newcommand{\DMVVG}{\textsc{Dynamic Maximum Value Vertex Guard}}



\title{Maximizing the Guarded Boundary of a Dynamic Art Gallery}
%\date{}

\author{Mihir Patel}



%%%%%%%%%%%%%%%%%%%%%%%%%%%%%%%%%%%%%%%%%%%%%%%%%%%%%%
\begin{document}

\maketitle

%mandatory: add short abstract of the document
\begin{abstract}
    An abstract.
\end{abstract}

% short version main body
\section{Introduction}
We consider a variant of the classic Art Gallery problem, where we instead seek to optimize the length of the boundary seen by the guards, not the number of guards themselves. Specifically, given a simple polygon $P$ and $k\in\mathbb{N}$, we want to find the $k$ vertex guards which maximize the length of the boundary of $P$ that is watched. We define the set of vertices of $P$ as $V_P$, and $L(S)$ as the length of the boundary seen by the set of vertex guards/vertices $S$. Note that $L(S)$ is necessarily at most the perimeter of $P$. 

\optproblemdef{\MLVG}{A simple polygon $P$ and a positive integer $k\in\mathbb{N}$.}{Find a set of vertices $S\subseteq V_P$ of size at most $k$ such that $L(S)$ is maximized.}

We also consider a weighted variant of this problem, where $P$ is a simple polygon composed of (possibly collinear) weighted line segments. Art galleries contain paintings, and some have more value than others. With limited guards, a realistic task would be to maximize not the total boundary watched, but the total value of paintings watched. In our definition, a weighted segment must be completely seen by our $k$ guards to be considered ``watched''. We define $W(S)$ as the weighted analog of $L(S)$, the sum of weights of all segments on the boundary of $P$ that are completely watched by guards in $S$.

\optproblemdef{\MVVG}{A weighted polygon $P$ and a positive integer $k\in\mathbb{N}$.}{Find a set of vertices $S\subseteq V_P$ of size at most $k$ such that $W(S)$ is maximized.}

Finally, we consider a dynamic version of \MVVG, which we'll call \DMVVG. New paintings may arrive in an art gallery, and their arrangement around the gallery may change. Thinking of vertex guards as cameras, it can be costly and time-consuming rearrange and reinstall cameras across the room (at least compared to moving the paintings around). If the segment weights on the weighted polygon change, how can we modify our solution (with minimal changes) to maintain an approximate solution for \MVVG\ at each timestep?
\todo[inline]{Obviously this dynamic problem needs to be specified a lot more, but hopefully the general idea of/motivation for the problem is showing through. I am trying to create a dynamic set cover analog for art gallery, where elements are inserted/removed one at a time, and the goal is to maintain a set cover that is still an approximate solution (without simply recalculating a set cover at each change).}


\subsection{Related Works}
\todo[inline]{Need a paragraph discussing more conventional Art gallery problem literature (authors+results). Klee, O'Rourke, Das}
In \cite{fragoudakis-interior,fragoudakis-boundary,fragoudakis-paintings}, Fragoudakis et al. pose both \MLVG\ and \MVVG. They prove that both problems are APX-complete in \cite{fragoudakis-boundary}, meaning these problems are NP-hard and also permit no PTAS unless $P=NP$. In \cite{fragoudakis-interior}, they present a $(1-1/e)$-approximation for maximizing the vertex-guarded \emph{interior} of a polygon which runs in $O(k^2n^2)$ and depends on segmenting the polygon into visibility regions.

\cite{abdelkader} presents several inapproximability results for art gallery problems, with $\alpha$-Floodlights.

\todo[inline]{Need to discuss some more dynamic set cover. Here is the most recent paper I've been able to find.}

\cite{bukov}

\subsection{Our Contributions}
For \MLVG\ and \MVVG, we extend the results from \cite{fragoudakis-interior} to get a $(1-1/e)$-approximation, that runs in $O(kn^2)$. We use the monotonicity and submodularity of our objective functions $L$ and $W$ to get this bound, presenting a simpler (and slightly more efficient) approach than the Finest Visbility Segmentation approach from \cite{fragoudakis-interior,fragoudakis-paintings}. We also prove several inapproximability results for these problems.

For \DMVVG, we analyze the performance of several strategies, showing some of them may perform arbitrarily badly compared to an optimal solution, while some achieve a bound. 









\section{Submodularity of \MVVG{}}
In this section, we prove that the objective functions of both \MLVG{} and \MVVG{} are monotone and submodular. This then implies that greedily maximizing the objective will yield a $(1-\frac{1}{e})$ approximation for both problems. We also provide a brief analysis of the running time of these greedy algorithms, which is an improvement on the approximation algorithms proposed for these problems in \cite{fragoudakis-interior,fragoudakis-boundary,fragoudakis-paintings}.\\\\
We start with the objective function of \MLVG{}, $L$. Recall that given a simple polygon $P$ and a set of vertex guards $S\subseteq V_P$, $L(S)$ denotes the length of the guarded boundary of $P$, and is necessarily less than or equal to the permiter of $P$. It is fairly immediate to see that $L$ is monotone.

\begin{observation}\label{obs:monotone}
    For any $S\subseteq T\subseteq V$, $L(S)\leq L(T)$.
\end{observation}

\begin{proof}
    Consider a point $p$ on the portion of the boundary of the polygon counted by $L(S)$, then $p$ must be seen by some $v\in S$. As $S\subseteq T$, $v\in T$ as well, implying $p$ is counted by $L(T)$. Every point on the boundary counted by $L(S)$ is also counted by $L(T)$, and thus $L(S)\leq L(T)$. 
\end{proof}
\noindent
We can also show that $L$ is submodular with a little bit more work.

\begin{claim}\label{clm:submodular}
    For any $S\subseteq T\subseteq V$ and $v\notin T$, $L(S\cup\{v\})-L(S)\geq L(T\cup\{v\})-L(T)$.
\end{claim}

\begin{proof}
    First observe that, by~\Cref{obs:monotone}, $L(S\cup\{v\})-L(S)\geq 0$, meaning this claim is trivial if $L(T\cup\{v\})-L(T)\leq 0$. So we only consider the case where $L(T\cup\{v\})-L(T)>0$, or in other words, there is some portion of the boundary of the polygon guarded by $T\cup\{v\}$, but not $T$. Then for every point $p$ in this portion, $p$ is visible to $v$, but not $T$. As $S\subseteq T$, by~\Cref{obs:monotone}, $p$ is visible to $S$ either. \\\\
    In short, we know that $p$ is visible to $S\cup\{v\}$ but not $S$, and as this is true for every point on the boundary seen by the set of guards $T\cup\{v\}$, but not $T$, $L(S\cup\{v\})-L(S)\geq L(T\cup\{v\})-L(T)$, as desired.
\end{proof}
\noindent
These two facts also lift quickly to the objective function of \MVVG{}, $W$. Given a simple polygon $P$ made up of weighted segments and a set of vertex guards $S\subseteq V_P$, $W(S)$ denotes the total weight of the guarded boundary of $P$. Note that a weighted segment must be \emph{completely} visible to a set of guards to be considered ``guarded'' and contribute to the total weight. The proofs of monotonicity and submodularity for $W$ proceed exactly the same as those of $L$ (we only need to swap $L$ and $W$).

\begin{observation}\label{obs:monotone-w}
    For any $S\subseteq T\subseteq V$, $W(S)\leq W(T)$.
\end{observation}

\begin{claim}\label{clm:submodular-w}
    For any $S\subseteq T\subseteq V$ and $v\notin T$, $W(S\cup\{v\})-W(S)\geq W(T\cup\{v\})-W(T)$.
\end{claim}
\noindent
Equipped with these properties of $L$ and $W$, we can quickly derive greedy algorithms for \MLVG{} and \MVVG{}. Start with an empty solution set $S$. While the size of $S$ is less than $k$, find the vertex $v$ which provides $L$ with the most marginal gain. That is, the vertex $v$ which maximizes $L(S\cup\{v\})-L(S)$. Then add $v$ to $S$ and repeat, until $|S|=k$. Then, because $L$ and $W$ are both monotone submodular, we get the following facts about maximizing monotone submodular functions under cardinality constraints (i.e. the type of problem \MLVG{} and \MVVG{} both are).

\begin{theorem}[\cite{cornuejols}]
    The greedy algorithm is a $(1-1/e)$-approximation algorithm for \MLVG{}. 
\end{theorem}

\begin{theorem}[\cite{feige}]
    For any $\epsilon>0$, there is no $(1-1/e+\epsilon)$-approximation algorithm for \MLVG{}, unless \cclass{P}=\cclass{NP}.
\end{theorem}
\noindent
The same approximation guarantees hold for the greedy algorithm applied to \MVVG{}, which maximizes the marginal gain of $W$. As for the runtime of these algorithms, they proceed in $k$ iterations, selecting one vertex per step. In each iteration, we evaluate the marginal gain for up to $n$ candidate vertices. To compute the marginal gain of a vertex $v$, we maintain the portion of the boundary currently guarded by the selected set $S$, compute the portion guarded by $v$, and take their union to determine the new coverage. Since computing the visibility region of a single vertex can be done in $O(n)$ time \cite{visibility}, and union operations can be managed efficiently with appropriate data structures, each marginal gain evaluation takes $O(n)$. Thus, the total running time of the greedy algorithm is $O(kn^2)$ in the worst case.\\\\
By leveraging the monotonicity and submodularity of the objective functions in \MLVG{} and \MVVG{}, we obtain simple greedy algorithms that achieve a $(1-1/e)$-approximation. While previous work \cite{fragoudakis-boundary,fragoudakis-interior,fragoudakis-paintings} also achieves this approximation ratio, we additionally prove that this is the best possible approximation ratio unless \cclass{P}=\cclass{NP}. Moreover, our approach offers a slight improvement in efficiency: our algorithms run in $O(kn^2)$ time, whereas their algorithms have runtimes of $O(n^4)$ and $O(k^2n^2)$.




\bibliographystyle{abbrvnat}
\bibliography{refs}


\end{document}
