\section{Appendix Section}
\begin{claim} \label{clm:incident}
    Given a graph $G=(V,E)$, $v\in V$, $k\in\mathbb{N}$, and a set of edges $S\subseteq V^2\backslash E$ of size at most $k$ where at least one edge in $S$ is not incident on $v$, there must always exist a set $T\subseteq V^2\backslash E$ such that $|T|\leq|S|$ and $c_{G+T}(v)<c_{G+S}(v)$.
\end{claim}
\begin{proof}
    Define the edge in $S$ not incident on $v$ as $xy$, using its endpoints $x,y\in V$. Without loss of generality, suppose $d_{G+S}(v,x)\leq d_{G+S}(v,y)$. For any vertex $z\in V$, we have two cases: either the shortest path from $v$ to $z$ in $G+S$ uses the edge $xy$ or it does not. In either case, with the edge set $T=(S\backslash\{xy\})\cup\{vy\}$, we will show that $d_{G+T}(v,z)\leq d_{G+S}(v,z)$.\\\\
    If the path from $v$ to $z$ in $G+S$ uses the edge $xy$, then $d_{G+S}(v,z)$ can be expressed as $d_{G+S}(v,x)+1+d_{G+S}(y,z)$. Note that the assumption about the relative distances of $x,y$ to $v$ guarantees that the shortest path from $v$ to $z$ crosses over $xy$ instead of $yx$, as $d_{G+S}(v,x)\leq d_{G+S}(v,y)<d_{G+S}(v,y)+d_{G+S}(y,x)$. Then in $G+T$,
    \[d_{G+T}(v,z)=d_{G+T}(v,y)+d_{G+T}(y,z)=1+d_{G+S}(y,z)\]\[<d_{G+S}(v,x)+1+d_{G+S}(y,z)=d_{G+S}(v,z).\]
    If the path from $v$ to $z$ in $G+S$ does not use the edge $xy$, then removing it and replacing it with a new edge cannot make the distance between $v$ and $z$ any greater, it can only decrease it. Thus in either case, $d_{G+T}(v,z)\leq d_{G+S}(v,z)$. As this is true for every vertex $z\in V$, and this inequality is strict when $z=y$, $c_{G+T}(v)<c_{G+S}(v)$, as desired.
\end{proof}

\begin{lemma} \label{lem:incident}
    Given a graph $G=(V,E)$, $v\in V$ and $k\in\mathbb{N}$, there is always a set $T\subseteq V^2\backslash E$ of at most $k$ edges, all incident on $v$, whose addition to the graph optimally improves the closeness centrality of $v$.  
\end{lemma}
\begin{proof}
    Assume not. That is, suppose the set $S$ of $k$ edges which optimally improved the closeness centrality of $v$ were not all incident on $v$. This means at least one edge in $S$ is not incident on $v$, implying the existence of some set $T$ of size at most $k$ such that $c_{G+T}(v)<c_{G+S}(v)$ (by Claim~\ref{clm:incident}), contradicting the assumption that $S$ optimally improved the closeness centrality of $v$.
\end{proof}

\begin{claim} \label{clm:noedges}
    Given a graph of the construction provided in Figure~\ref{cri-reduction} derived from a set cover instance with $m$ sets, $n$ elements and $k\in\mathbb{N}$, if there is no $k$-sized set cover, then there must exist some element vertex with no newly-added edge incident on it or any vertex of a set containing it.
\end{claim}
\begin{proof}
   Assume not. That is, every element vertex \textit{does} have a new edge incident on it or one of the set vertices containing it. For any element vertex that is the endpoint of a new edge, simply reconstruct this edge to one of the set vertices adjacent to this element. Now we have identified $k$ edges incident only on set vertices such that every element is the neighbor of at least one such set vertex, and thus we have identified a $k$-sized set cover (a contradiction). 
\end{proof}

\begin{claim} \label{clm:set-incident}
    Given a graph of the construction provided in Figure~\ref{cri-reduction} derived from a set cover instance of $m$ sets, $n$ elements and $k\in\mathbb{N}$ with no $k$-sized set cover, given a set of at most $k$ edges $S\subseteq V^2\backslash E$, all incident on $b\in V$, if at least one edge is not incident on a set vertex, then there must exist a set $T\subseteq V^2\backslash E$ such that $|T|=|S|$ and $c_{G+T}(b)\leq c_{G+S}(b)$.
\end{claim}

\begin{proof}
    Suppose $S\subseteq V^2\backslash E$ is a set of $k$ edges which  contains only edges incident on $b$, but at least one edge is not incident on a set vertex. Given $b$ is connected to $a$ and $c$, this implies that there must exist an edge in $S$ which is connected a vertex $x_i$ in the independent set or an element vertex $e_i$. \\\\
    In the first case, this edge reduces the closeness centrality of $b$ by at most 1 --- it has changed $d(x_i,b)$ from 2 to 1, but not made $b$ closer to any other vertices. If we instead connect $b$ to a set vertex $s_i$, then $d(s_i,b)$ has been changed from 2 to 1, meaning this edge reduces the closeness centrality of $b$ by at least as much as the edge $b-x_i$. \\\\
    If the edge not incident on a set vertex instead connects $b$ to an element vertex $e_i$, this edge reduces the closeness centrality of $b$ by at most 2 --- it has changed $d(e_i,b)$ from 3 to 1, but not made $b$ closer to any other vertices. As there is no $k$-sized set cover, Claim~\ref{clm:noedges} tells us there must be some element vertex $e_j$ with no new edges incident on it or any set containing it. So instead connect $b$ to a set vertex $s_j$, where $e_j\in S_j$. then both $d(s_j,b)$ and $d(e_j,b)$ have been changed from 2 to 1, meaning $b-s_j$ reduces the closeness centrality of $b$ by at least as much as the edge $b-e_i$. \\\\ 
    If we define the set $T$ as the set $S$, but with the above case-dependent substitutions made, then we have that $|T|=|S|$ and $c_{G+T}(b)\leq c_{G+S}(b)$, as desired.
\end{proof}

\begin{lemma} \label{lem:set-incident}
    Given a graph of the construction provided in Figure~\ref{cri-reduction} derived from a set cover instance of $m$ sets, $n$ elements and $k\in\mathbb{N}$ with no $k$-sized set cover, there is always a set $T\subseteq V^2\backslash E$ of at most $k$ edges, all incident on both $b\in V$ and any set vertex, whose addition to the graph optimally improves the closeness centrality of $b$.
\end{lemma}

\begin{proof}
    From Lemma~\ref{lem:incident}, we know that there exists a set $T$ of $k$ edges which optimally improves the closeness centrality of $b$ such that all edges in $T$ are incident on $b$ itself. What remains to be shown is that there exists such a set of edges where they are all incident on $b$ \textit{and} some set vertex $s_i$.\\\\
    Assume not. That is, all the sets of edges which optimally improve $b$'s closeness centrality contain only edges incident on $b$, but at least one edge not incident on a set vertex. Now consider one such set of edges $S$. Claim~\ref{clm:incident} gives us a set $S'$ with $c_{G+S'}(b)\leq c_{G+S}(b)$. If $S'$ still contains an edge incident on $b$, but not a set vertex, we can reapply Claim~\ref{clm:incident} to get a new set that improves $b$'s closeness centrality just as well. In fact, we can keep applying this claim until we have a set of $k$ edges $T$ such that $|T|=|S|$ and $c_{G+T}(b)\leq c_{G+S}(b)$ and all edges in $T$ are incident on both $b$ and any set vertex. However, we assumed that all the sets of edges which optimally improve $b$'s closeness centrality contain only edges incident on $b$, but must have at least one edge not incident on a set vertex, a contradiction. 
\end{proof}

\begin{lemma}\label{lem:integer}
    For $m,n,k\in\mathbb{N}$ and $\tau\in\left(\frac{1}{2},1\right]$, there exists an\\ $X\in\left(\frac{2+2m+3n-\tau(2+2m-k+2n+1)}{2\tau-1},\frac{2+2m+3n-\tau(2+2m-k+2n)}{2\tau-1}\right]$ such that $X\in\mathbb{N}$.
\end{lemma}
\begin{proof}
    Taking the difference of the bounds of the interval\\ $\left(\frac{2+2m+3n-\tau(2+2m-k+2n+1)}{2\tau-1},\frac{2+2m+3n-\tau(2+2m-k+2n)}{2\tau-1}\right]$, we get that the length of the interval is $\frac{\tau}{2\tau-1}$, which is greater than or equal to $1$ for all $\tau\in\left(\frac{1}{2},1\right]$. Therefore, the length of our interval is bounded below by 1, and so there must exist some integer $X$ within this interval. Furthermore, as the numerator and denominator of the bounds of this interval are positive when $m,n,k\in\mathbb{N}$ and $\tau\in\left(\frac{1}{2},1\right]$, $X\in\mathbb{N}$ as desired.
\end{proof}


\begin{theorem} \label{t2-hard}
    Closeness Ratio Improvement is NP-Hard for $\tau\in(\frac{1}{2},1)$.
\end{theorem}
\begin{proof}
    (Concluding the argument started earlier, with construction given).\\\\
    \textbf{$k$-sized set cover $\Longrightarrow$ $\frac{\min(c_{G+T}(a),c_{G+T}(b))}{\max(c_{G+T}(a),c_{G+T}(b))}\geq\tau$.} If there is a set cover, follow the same protocol as in the proof of Theorem~\ref{t1-hard} (connect $b$ to the $k$ set vertices representing sets in the cover). In this case, $c_{G+T}(a)=X+2+2m+3n$ and $c_{G+T}(b)=2X+2+2m-k+2n$. \\\\
    We want to show that the ratio we achieve is greater than or equal to $\tau$, but we must still ensure that our ratio is less than or equal to 1, else we violate our min/max definition of closeness ratio. Note that this was not a conceren when $\tau=1$, because we showed our ratio when there was a set cover was $1$ exactly. So we consider the case where $c_{G+T}(b)\geq c_{G+T}(a)$ and the case where $c_{G+T}(b)<c_{G+T}(a)$, and show that in both cases the ratio of $\frac{\min(c_{G+T}(a),c_{G+T}(b))}{\max(c_{G+T}(a),c_{G+T}(b))}\geq\tau$.\\\\
    If $c_{G+T}(b)\geq c_{G+T}(a)$, then
    \[\frac{\min(c_{G+T}(a),c_{G+T}(b))}{\max(c_{G+T}(a),c_{G+T}(b))}=\frac{X+2+2m+3n}{2X+2+2m-k+2n}\]\[\geq\frac{(\frac{2+2m+3n-\tau(2+2m-k+2n)}{2\tau-1})+2+2m+3n}{2(\frac{2+2m+3n-\tau(2+2m-k+2n)}{2\tau-1})+2+2m-k+2n}\geq\tau\]
    Alternatively, if $c_{G+T}(b)<c_{G+T}(a)$, then
    \[\frac{\min(c_{G+T}(a),c_{G+T}(b))}{\max(c_{G+T}(a),c_{G+T}(b))}=\frac{2X+2+2m-k+2n}{X+2+2m+3n}\]\[>\frac{2(\frac{2+2m+3n-\tau(2+2m-k+2n+1)}{2\tau-1})+2+2m-k+2n}{(\frac{2+2m+3n-\tau(2+2m-k+2n+1)}{2\tau-1})+2+2m+3n}=\frac{(2-2\tau)+2m+4n+k}{\tau+2m\tau+4n\tau+k\tau}\]
    As $\tau\in(\frac{1}{2},1)$, $2-2\tau>0$, implying:
    \[\frac{(2-2\tau)+2m+4n+k}{\tau+2m\tau+4n\tau+k\tau}>\frac{2m+4n+k}{\tau+2m\tau+4n\tau+k\tau}=\frac{1}{\tau}(\frac{2m+4n+k}{1+2m+4n+k})\]
    We chose our Set Cover instance such that $\frac{2m+4n+k}{1+2m+4n+k}\geq\tau^2$. Thus we have that:
    \[\frac{1}{\tau}(\frac{2m+4n+k}{1+2m+4n+k})\geq\frac{1}{\tau}(\tau^2)=\tau\]
    Thus in either case, if there is a set cover of $U$, we can add $k$ edges to $G$ to get a closeness ratio greater than or equal to $\tau$.\\\\
    \textbf{No $k$-sized set cover $\Longrightarrow$ $\frac{\min(c_{G+T}(a),c_{G+T}(b))}{\max(c_{G+T}(a),c_{G+T}(b))}<\tau$.} If there is no $k$-sized set cover, we use the same analysis from the proof of Theorem~\ref{t1-hard} to argue that the best way $b$ can reduce its closeness centrality is by connecting itself to set vertices. Furthermore, this method makes $k$ set vertices distance 1 from $b$, and thus $n-1$ element vertices distance 2 from $b$, but (using Claim~\ref{clm:noedges}) there must be some element which remains distance 3 from $b$. Then the closeness ratio of $a,b$ is:
    \[\frac{\min(c_{G+T}(a),c_{G+T}(b))}{\max(c_{G+T}(a),c_{G+T}(b))}=\frac{X+2+2m+3n}{2X+2+2m-k+2n+1}\]\[<\frac{(\frac{2+2m+3n-\tau(2+2m-k+2n+1)}{2\tau-1})+2+2m+3n}{2(\frac{2+2m+3n-\tau(2+2m-k+2n+1)}{2\tau-1})+2+2m-k+2n+1}=\tau\]
\end{proof}