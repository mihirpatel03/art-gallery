\section{Introduction}
We consider a variant of the classic Art Gallery problem, where we instead seek to optimize the length of the boundary seen by the guards, not the number of guards themselves. Specifically, given a simple polygon $P$ and $k\in\mathbb{N}$, we want to find the $k$ vertex guards which maximize the length of the boundary of $P$ that is watched. We define the set of vertices of $P$ as $V_P$, and $L(S)$ as the length of the boundary seen by the set of vertex guards/vertices $S$. Note that $L(S)$ is necessarily at most the perimeter of $P$. 

\optproblemdef{\MLVG}{A simple polygon $P$ and a positive integer $k\in\mathbb{N}$.}{Find a set of vertices $S\subseteq V_P$ of size at most $k$ such that $L(S)$ is maximized.}

We also consider a weighted variant of this problem, where $P$ is a simple polygon composed of (possibly collinear) weighted line segments. Art galleries contain paintings, and some have more value than others. With limited guards, a realistic task would be to maximize not the total boundary watched, but the total value of paintings watched. In our definition, a weighted segment must be completely seen by our $k$ guards to be considered ``watched''. We define $W(S)$ as the weighted analog of $L(S)$, the sum of weights of all segments on the boundary of $P$ that are completely watched by guards in $S$.

\optproblemdef{\MVVG}{A weighted polygon $P$ and a positive integer $k\in\mathbb{N}$.}{Find a set of vertices $S\subseteq V_P$ of size at most $k$ such that $W(S)$ is maximized.}

Finally, we consider a dynamic version of \MVVG, which we'll call \DMVVG. New paintings may arrive in an art gallery, and their arrangement around the gallery may change. Thinking of vertex guards as cameras, it can be costly and time-consuming rearrange and reinstall cameras across the room (at least compared to moving the paintings around). If the segment weights on the weighted polygon change, how can we modify our solution (with minimal changes) to maintain an approximate solution for \MVVG\ at each timestep?
\todo[inline]{Obviously this dynamic problem needs to be specified a lot more, but hopefully the general idea of/motivation for the problem is showing through. I am trying to create a dynamic set cover analog for art gallery, where elements are inserted/removed one at a time, and the goal is to maintain a set cover that is still an approximate solution (without simply recalculating a set cover at each change).}


\subsection{Related Works}
\todo[inline]{Need a paragraph discussing more conventional Art gallery problem literature (authors+results). Klee, O'Rourke, Das}
In \cite{fragoudakis-interior,fragoudakis-boundary,fragoudakis-paintings}, Fragoudakis et al. pose both \MLVG\ and \MVVG. They prove that both problems are APX-complete in \cite{fragoudakis-boundary}, meaning these problems are NP-hard and also permit no PTAS unless $P=NP$. In \cite{fragoudakis-interior}, they present a $(1-1/e)$-approximation for maximizing the vertex-guarded \emph{interior} of a polygon which runs in $O(k^2n^2)$ and depends on segmenting the polygon into visibility regions.

\cite{abdelkader} presents several inapproximability results for art gallery problems, with $\alpha$-Floodlights.

\todo[inline]{Need to discuss some more dynamic set cover. Here is the most recent paper I've been able to find.}

\cite{bukov}

\subsection{Our Contributions}
For \MLVG\ and \MVVG, we extend the results from \cite{fragoudakis-interior} to get a $(1-1/e)$-approximation, that runs in $O(kn^2)$. We use the monotonicity and submodularity of our objective functions $L$ and $W$ to get this bound, presenting a simpler (and slightly more efficient) approach than the Finest Visbility Segmentation approach from \cite{fragoudakis-interior,fragoudakis-paintings}. We also prove several inapproximability results for these problems.

For \DMVVG, we analyze the performance of several strategies, showing some of them may perform arbitrarily badly compared to an optimal solution, while some achieve a bound. 








