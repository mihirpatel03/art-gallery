\section{Intuition-Building Examples for \BMVVG{}}
In this section, we show that three natural greedy strategies for \BMVVG{} may all perform arbitrarily badly compared to an optimal solution. In other words, these strategies will not automatically form approximations for \BMVVG{}, as there exists a problem instance where the total weight they achieve is \emph{not} boundedly close to the optimal. Note that this is in contrast to \MLVG{} and \MVVG{}, where the natural greedy strategy formed a constant-factor approximation, validating the added difficulty of \BMVVG{}. \\\\
A first attempt could be to greedily add the vertex which maximizes the total weight guarded. To see why this fails, fix a $w\in\mathbb{N}$ such that $w\geq 3$. Then construct the polygon depicted in~\Cref{fig:greedy-heaviest}, with $w$ many ``cones''. The right edge of each cone has weight $w$, while all other edges have weight 1. The apex of each cone has cost $w$, while all other vertices have cost $w^2$. The budget $B$ is equal to $w^2$, and now we have an instance of \BMVVG{}, with~\Cref{fig:greedy-heaviest} and $B=w^2$.
\begin{figure}
    \centering
    \begin{tikzpicture}
        %cone 1
        \coordinate (a) at (0, -.25);
        \coordinate (b) at (1, 1.5);
        \coordinate (c) at (2, 0);
        %cone 2
        \coordinate (d) at (3, 1.5);
        \coordinate (e) at (4, 0);
        %cone 3
        \coordinate (f) at (5, 1.5);
        \coordinate (g) at (6, 0);

        %cone 4
        \coordinate (h) at (7.5, 0);
        \coordinate (i) at (8.5, 1.5);
        \coordinate (j) at (9.5, -.25);

        % Draw the polygon
        \draw[thick] (a) -- (b) -- (c) -- (d) -- (e) -- (f) -- (g);
        \draw[thick] (h) -- (i) -- (j) -- (a);

        % Add dots at each vertex
        \fill (a) circle (2pt);
        \fill (b) circle (2pt);
        \fill (c) circle (2pt);
        \fill (d) circle (2pt);
        \fill (e) circle (2pt);
        \fill (f) circle (2pt);
        \fill (g) circle (2pt);
        \fill (h) circle (2pt);
        \fill (i) circle (2pt);
        \fill (j) circle (2pt);

        %ellipsis
        \node at (6.8, .75) {\huge$\cdots$};

        % Add edge weights 
        \node[left, text=blue] at ($(a)!0.75!(b)$) {$1$};
        \node[right, text=blue] at ($(b)!0.3!(c)$) {$w$};

        \node[left, text=blue] at ($(c)!0.75!(d)$) {$1$};
        \node[right, text=blue] at ($(d)!0.3!(e)$) {$w$};

        \node[left, text=blue] at ($(e)!0.75!(f)$) {$1$};
        \node[right, text=blue] at ($(f)!0.3!(g)$) {$w$};

        \node[left, text=blue] at ($(h)!0.75!(i)$) {$1$};
        \node[right, text=blue] at ($(i)!0.25!(j)$) {$w$};

        \node[below, text=blue] at ($(a)!0.5!(j)$) {$1$};

    \end{tikzpicture}
    \caption{apex vertices have cost w, everything else has $w^2$, budget is $w^2$.}
    \label{fig:greedy-heaviest}
\end{figure}\\\\
Now the greedy strategy would start by choosing the vertex that maximizes the total weight guarded. Initially, this is one of the ``valleys'' in polygon, as these vertices guard $2w+2$ total weight, whereas all other vertices guard at most $w+1$. Any one of these vertices has cost $w^2$, so adding it to our solution uses our entire budget. Thus the algorithm which greedily maximizes total weight guarded ends up guarding $2w+2$ total weight on~\Cref{fig:greedy-heaviest}.\\\\
However, an optimal strategy would be to place a guard at each apex. There are $w$ many apices and each one has cost $w$, so to use all of them would cost $w^2$ (which is exactly our budget). This guard placement also guards our entire polygon boundary, meaning the total weight guarded by the optimal set of guards is $w(w)+w(1)+1=w^2+w+1$. To compare total weight achieved between these two algorithms, the weight achieved by the greedy divided by the weight achieved by the optimal is equal to $\frac{2w+2}{w^2+w+1}$. We can make this value arbitrarily small as we increase $w$, meaning the greedy algorithm which maximizes total weight guarded performs arbitrarily badly compared to an optimal solution.\\\\
Now we consider two alternative greedy strategies and show that they also fail, using~\Cref{fig:greedy-cheapest}. The first would be to greedily add the vertex with cheapest cost. The second would be to greedily add the vertex which maximizes the ratio of marginal gain in total weight to vertex cost. Starting with the first strategy, consider~\Cref{fig:greedy-cheapest} where weights and costs are displayed, and the budget is $w+4$. The cheapest vertex is $A$, which has cost of $2$. The vertex $A$ guards total weight of $2$, and once we have placed a guard there we cannot afford to place guards at any of the remaining vertices. Thus this greedy choosing the cheapest vertex achieves total weight of $2$ on this example. An optimal strategy would be to place a guard on either $D$ or $B$, this stays within budget and guards the entire polygon, achieving total weight of $w+3$. Comparing the weight achieved by each strategy, we have $\frac{2}{w+3}$, which we can make arbitrarily small as we increase $w$.
\begin{figure}
    \centering
    \begin{tikzpicture}[scale=4]
        % Define the points of the reflex quadrilateral
        \coordinate (A) at (0, 0);
        \coordinate (B) at (1, 0);
        \coordinate (C) at (0.6, 1);
        \coordinate (D) at (.53, .33); % Reflex angle at D

        % Draw the polygon
        \draw[thick] (A) -- (B) -- (C) -- (D) -- cycle;

        % Draw vertex dots
        \foreach \point in {A,B,C,D}
            \fill (\point) circle (0.015);

        % Label the vertices
        \node[below left] at (A) {$A$};
        \node[below right] at (B) {$B$};
        \node[above] at (C) {$C$};
        \node[above left] at (D) {$D$};

        % Add edge weights (all same color)
        \node[below, text=blue] at ($(A)!0.5!(B)$) {$1$};
        \node[right, text=blue] at ($(B)!0.5!(C)$) {$1$};
        \node[above left, text=blue] at ($(C)!0.5!(D)$) {$w$}; % edge with weight w
        \node[above left, text=blue] at ($(D)!0.5!(A)$) {$1$};

        % Compute average y-coordinate for alignment
        \path let \p1 = (A), \p2 = (C) in
            coordinate (tableanchor) at ($(0, {(\y1 + \y2)/2})$);

        % Add cost table to the right, vertically aligned with polygon
        \node[anchor=west] at ($(B) + (.5cm, 0.5cm)$) {
            \begin{tabular}{c|c}
            Vertex & Cost \\
            \hline
            $A$ & $2$ \\
            $B$ & $w + 4$ \\
            $C$ & $w + 4$ \\
            $D$ & $w + 4$ \\
            \end{tabular}
        };

    \end{tikzpicture}
    \caption{caption}
    \label{fig:greedy-cheapest}
\end{figure}\\\\
Similarly, if we wanted to greedily choose the vertex which maximizes the ratio of marginal gain in total weight to vertex cost (i.e., provides the best ``bang-for-your-buck''), we would choose vertex $A$. It guards weight of $2$ and costs $2$, giving a ratio of $1$. Any other vertex can guard at most $w+3$, while costing $w+4$, and $\frac{w+3}{w+4}<1$ for all $w$. Again, choosing $A$ prevents adding any more guards, achieving a total guarded weight of $2$, while an optimal strategy could guard $w+3$, and $\frac{2}{w+3}$ can be arbitrarily small for large $w$. \\\\
\cite{khuller} considers an analagous problem of covering as many elements as possible with sets, subject to a budget. They similarly show that greedily adding the set which maximizes the ratio of marginal gain in total weight to set cost does not form an approximation. However, they show that if you take the collection of sets $\mathcal{S}$ obtained by this greedy strategy, and you also take the single set which covers the most weight $S_t$, and then you output whichever one covers more weight, you get a $\frac{1}{2}(1-1/e)$-approximation for their problem. And they also show that this simple strategy can be further refined to obtain a $(1-1/e)$-approximation. A logical next step for \BMVVG{} would be to see how well these strategies can do. That is, if we take the set of vertices which was greedily chosen by maximizing the ratio of marginal gain in total weight guarded to vertex cost, and we take the vertex which guards the most weight alone, and we return whichever set of guards ends up guarding more weight, does this give us an approximation? Is it $\frac{1}{2}(1-1/e)$?




