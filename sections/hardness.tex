\section{Hardness of Closeness Ratio Improvement}
In this section, we establish the NP-hardness of CRI, over a range of target ratios.

\begin{theorem}
    Closeness Ratio Improvement is NP-Hard for $\tau=1$
\end{theorem}

\begin{proof} 
    Given an instance of \textsc{Set Cover} with universe $U=\{e_1,e_2,\ldots,e_n\}$, a collection of non-empty subsets $S_1,S_2,\ldots,S_m\subseteq U$, and a positive integer $k\in\mathbb{N}$, construct a decision instance of \textsc{Closeness Ratio Improvement} as follows. Vertices $a,b,c$ form a clique. Each of the $n$ elements from the universe have a corresponding vertex $e_1,e_2,\ldots,e_n$. For each set $S_i$, create vertex $s_i$ and connect it to $c$. Next, connect $s_i$ to each vertex that represents an element contained within $S_i$. Finally, create an independent set of $X=n+k$ vertices each connected to $a$. The construction of $G$ is depicted in Figure~\ref{cri-instance}. \\\\
    We claim that there exists a $k$-sized set cover of $U$ if and only if there is a set $T$ of at most $k$ edges such that $\frac{\min(c_{G+T}(a),c_{G+T}(b))}{\max(c_{G+T}(a),c_{G+T}(b))}\geq\tau=1$. Note that in our construction, $a$ is initially more central than $b$. As we cannot increase the closeness centrality of a vertex by adding edges, the task of making their ratio closer to 1 is initially analagous to decreasing $b$'s centrality.
    \begin{figure} \label{cri-instance}
    \centering
    \begin{tikzpicture}[ 
        roundnode/.style={circle, draw=black, fill =white, minimum size=8mm},
        largenode/.style={circle, draw=black, minimum size=14mm},
        goldnode/.style={circle, draw=black, fill=red, minimum size=8mm}
        bluenode/.style={circle, draw=black, fill=red, minimum size=8mm},
    ]

    % Nodes
    \node[roundnode] (a) at (-1.5,3) {$a$};
    \node[roundnode] (b) at (1.5,3) {$b$};
    \node[roundnode] (c) at (0,1) {$c$};

    \node[largenode] (is) at (-3.5,3) {$IS_X$};

    \node[roundnode] (S1) at (-2,-0.75) {$s_1$};
    \node[roundnode] (S2) at (-.5,-0.75) {$s_2$};
    \node (dots) at (.75,-0.75) {$\cdots$};
    \node[roundnode] (Sm) at (2,-0.75) {$s_m$};

    \node[roundnode] (E1) at (-3,-3) {$e_1$};
    \node[roundnode] (E2) at (-1.5,-3) {$e_2$};
    \node[roundnode] (E3) at (0,-3) {$e_3$};
    \node (dotsE) at (1.5,-3) {$\cdots$};
    \node[roundnode] (En) at (3,-3) {$e_n$};

    % Edges
    \draw[-] (a) -- (b);
    \draw[-] (a) -- (c);
    \draw[-] (b) -- (c);

    \draw[-] (c) -- (S1);
    \draw[-] (c) -- (S2);
    \draw[-] (c) -- (Sm);

    \draw[-] (S1) -- (E1);
    \draw[-] (S1) -- (E2);
    \draw[-] (Sm) -- (E3);
    \draw[-] (S2) -- (E2);
    \draw[-] (Sm) -- (E1);
    \draw[-] (Sm) -- (En);

    \draw[-] (a) -- ($(is.east) + (0,0.3)$);

    % Second edge from 'a' to the east side of 'IS_X' slightly below center
    \draw[-] (a) -- ($(is.east) + (0,-0.3)$);
    \end{tikzpicture}
    \caption{Construction of a CRI instance with vertices $a,b$ from Set Cover.}
    \end{figure}\\\\
    \textbf{$k$-sized set cover $\Longrightarrow$ $\frac{\min(c_{G+T}(a),c_{G+T}(b))}{\max(c_{G+T}(a),c_{G+T}(b))}=1$.} For each set $S_i$ in the cover, construct the edge $bs_i$. Now every element vertex must be adjacent to a set vertex in the cover (by definition of a cover), and then each of these set vertices has a newly constructed edge to $b$. Thus each element vertex is distance 2 from $b$, and $k$ set vertices are distance 1 away from $b$. Now $c_{G+T}(b)=2(n+k)+1(2)+1(k)+2(m-k)+2(n)=4n+2m+k+2$. Similarly, $c_{G+T}(a)=1(n+k)+1(2)+2(m)+3(n)=4n+2m+k+2$. Then the ratio of $a$ and $b$'s closeness centrality must be 1, as desired.\\\\
    \textbf{No $k$-sized set cover $\Longrightarrow$ $\frac{\min(c_{G+T}(a),c_{G+T}(b))}{\max(c_{G+T}(a),c_{G+T}(b))}<1$.} As previously stated, in order to maximize the closeness ratio of $a$ and $b$, we decrease $b$'s closeness centrality as much as possible with $k$ edges.\\\\
    \begin{lemma}\ref{lem:set-incident}
        In this construction, if there is no $k$-sized set cover, there is always a set $T\subseteq V^2\backslash E$ of at most $k$ edges, all incident on both $b\in V$ and any set vertex, whose addition to the graph optimally improves the closeness centrality of $b$.
    \end{lemma}
    \begin{proof}
        Deferred to appendix.
    \end{proof}
    \noindent
    In other words, the set of edges which will optimally decrease $b$'s closeness centrality are all incident on $b$ and some set vertex. Therefore, the best $b$ can reduce its closeness centrality is by connecting itself to set vertices. This method can make $k$ set vertices distance 1 from $b$, and thus at most $n-1$ element vertices distance 2 from $b$, as there is no cover. 

    \begin{claim}
        In this construction, if there is no $k$-sized set cover, then there must exist some element vertex with no newly-added edge incident on it or any vertex of a set containing it.
    \end{claim}
    \begin{proof}
        Deferred to appendix.
    \end{proof}
    \noindent
    However, using the above claim, there must be some element which remains distance 3 from $b$. Then the closeness ratio of $a,b$ is:
    \[\frac{\min(c_{G+T}(a),c_{G+T}(b))}{\max(c_{G+T}(a),c_{G+T}(b))}<\frac{(n+k)+2+2m+3n}{2(n+k)+2+2m-k+2n+1}=\frac{4n+2m+k+2}{4n+2m+k+3}<1\]
\end{proof}
\noindent
We have shown that it is NP-hard to achieve a closeness centrality ratio of 1, but are smaller ratios achievable in polynomial time? By manipulating the size of the independent set connected to $a$ (for $\tau=1$, it was $n+k$ vertices) we can in fact prove a much stronger hardness result. 

\begin{theorem}
    Closeness Ratio Improvement is NP-Hard for $\tau\in(\frac{1}{2},1)$.
\end{theorem}
\noindent
We will go through the construction of a Closeness Ratio Improvement instance from Set Cover, but the analysis of yes and no cases is deferred to the appendix.
\begin{proof}
    Fix an arbitrary $\tau\in(\frac{1}{2},1)$. Consider an instance of Set Cover with $m$ sets, $n$ elements, and $k\in\mathbb{N}$ which satisfies $\frac{2m+4n+k}{1+2m+4n+k}\geq\tau^2$. This fraction converges to $1$ as we increase $m,n,k$, and $\tau^2<1$ when $\tau\in(\frac{1}{2},1)$, so we should always be able to find $m,n,k$ that satisfy this inequality. Use the same construction of $G$ described in the proof of Theorem  and depicted in Figure. However, now let $X$ (the number of vertices in the independent set attached to $a\in V$) be an integer in the interval $(\frac{2+2m+3n-\tau(2+2m-k+2n+1)}{2\tau-1},$\\$\frac{2+2m+3n-\tau(2+2m-k+2n)}{2\tau-1}]$. We know such an integer always exists by Lemma. \\\\
    We claim that there exists a $k$-sized set cover of $U$ if and only if there is a set $T$ of at most $k$ edges such that $\frac{\min(c_{G+T}(a),c_{G+T}(b))}{\max(c_{G+T}(a),c_{G+T}(b))}\geq\tau$.
\end{proof}



%%-------------------------------------------------------------------------------------------------
%% CUT STOP: preserve the command below to generate the reference list

