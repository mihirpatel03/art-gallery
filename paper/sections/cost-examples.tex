\section{Intuition-Building Examples for \BMVVG{}}
In this section, we show that three natural greedy strategies for \BMVVG{} may all perform arbitrarily badly compared to an optimal solution. In other words, these strategies will not automatically form approximations for \BMVVG{}, as there exists a problem instance where the total weight they achieve is \emph{not} boundedly close to the optimal. Note that this is in contrast to \MLVG{} and \MVVG{}, where the natural greedy strategy formed a constant-factor approximation, validating the added difficulty of \BMVVG{}. \\\\
A first attempt could be to greedily add the vertex which maximizes the marginal gain in total weight guarded. To see why this fails, fix a $w\in\mathbb{N}$ such that $w\geq 3$. Then construct the polygon depicted in~\Cref{fig:greedy-heaviest}, with $w$ many ``cones''. The right edge of each cone has weight $w$, while all other edges have weight 1. The apex of each cone has cost $w$, while all other vertices have cost $w^2$. The budget $B$ is equal to $w^2$, and now we have an instance of \BMVVG{}, with~\Cref{fig:greedy-heaviest} and $B=w^2$.
\begin{figure}
    \centering
    \begin{tikzpicture}
        %cone 1
        \coordinate (a) at (0, -.25);
        \coordinate (b) at (1, 1.5);
        \coordinate (c) at (2, 0);
        %cone 2
        \coordinate (d) at (3, 1.5);
        \coordinate (e) at (4, 0);
        %cone 3
        \coordinate (f) at (5, 1.5);
        \coordinate (g) at (6, 0);

        %cone 4
        \coordinate (h) at (7.5, 0);
        \coordinate (i) at (8.5, 1.5);
        \coordinate (j) at (9.5, -.25);

        % Draw the polygon
        \draw[thick] (a) -- (b) -- (c) -- (d) -- (e) -- (f) -- (g);
        \draw[thick] (h) -- (i) -- (j) -- (a);

        % Add dots at each vertex
        \fill (a) circle (2pt);
        \fill (b) circle (2pt);
        \fill (c) circle (2pt);
        \fill (d) circle (2pt);
        \fill (e) circle (2pt);
        \fill (f) circle (2pt);
        \fill (g) circle (2pt);
        \fill (h) circle (2pt);
        \fill (i) circle (2pt);
        \fill (j) circle (2pt);

        %ellipsis
        \node at (6.8, .75) {\huge$\cdots$};

        % Add edge weights 
        \node[left, text=blue] at ($(a)!0.75!(b)$) {$1$};
        \node[right, text=blue] at ($(b)!0.3!(c)$) {$w$};

        \node[left, text=blue] at ($(c)!0.75!(d)$) {$1$};
        \node[right, text=blue] at ($(d)!0.3!(e)$) {$w$};

        \node[left, text=blue] at ($(e)!0.75!(f)$) {$1$};
        \node[right, text=blue] at ($(f)!0.3!(g)$) {$w$};

        \node[left, text=blue] at ($(h)!0.75!(i)$) {$1$};
        \node[right, text=blue] at ($(i)!0.25!(j)$) {$w$};

        \node[below, text=blue] at ($(a)!0.5!(j)$) {$1$};

    \end{tikzpicture}
    \caption{A polygon composed of $w$ cones, where the right edge of each cone has weight $w$ and all other edges in the polygon have weight 1. The apicies of each cone have cost $w$, and everything else has cost $w^2$. We are given a budget of $B=w^2$.}
    \label{fig:greedy-heaviest}
\end{figure}\\\\
The porposed greedy strategy would start by choosing the vertex that maximizes the total weight guarded. Initially, this is one of the ``valleys'' in the polygon, as these vertices guard $2w+2$ total weight, whereas all other vertices guard at most $w+1$. Any one of these vertices has cost $w^2$, so adding it to our solution uses our entire budget. Thus the algorithm which greedily maximizes the marginal gain in total weight guarded ends up guarding $2w+2$ total weight on~\Cref{fig:greedy-heaviest}.\\\\
However, an optimal strategy would be to place a guard at each apex. There are $w$ many apices and each one has cost $w$, so to use all of them would cost $w^2$ (which is exactly our budget). This guard placement also guards our entire polygon boundary, meaning the total weight guarded by the optimal set of guards is $w(w)+w(1)+1=w^2+w+1$. To compare the total weight achieved between these two algorithms, the weight achieved by the greedy divided by the weight achieved by the optimal is equal to $\frac{2w+2}{w^2+w+1}$. We can make this value arbitrarily small as we increase $w$, meaning the greedy algorithm which maximizes total weight guarded performs arbitrarily badly compared to an optimal solution and will not achieve any constant factor approximation ratio.\\\\
Now we consider two alternative greedy strategies and show that they also fail, using~\Cref{fig:greedy-cheapest}. The first would be to greedily add the vertex with cheapest cost. The second would be to greedily add the vertex which maximizes the ratio of marginal gain in total weight to vertex cost. Starting with the first strategy, consider~\Cref{fig:greedy-cheapest} where weights and costs are displayed, and the budget is $w+4$. The cheapest vertex is $A$, which has cost of $2$. The vertex $A$ guards total weight of $2$, and once we have placed a guard there, we cannot afford to place guards at any of the remaining vertices. Thus this greedy strategy achieves total weight of $2$ on this example. An optimal strategy would be to place a guard on either $D$ or $B$, this stays within budget and guards the entire polygon, achieving total weight of $w+3$. Comparing the weight achieved by each strategy, we have $\frac{2}{w+3}$, which we can again make arbitrarily small as we increase $w$.
\begin{figure}
    \centering
    \begin{tikzpicture}[scale=4]
        % Define the points of the reflex quadrilateral
        \coordinate (A) at (0, 0);
        \coordinate (B) at (1, 0);
        \coordinate (C) at (0.6, 1);
        \coordinate (D) at (.53, .33); % Reflex angle at D

        % Draw the polygon
        \draw[thick] (A) -- (B) -- (C) -- (D) -- cycle;

        % Draw vertex dots
        \foreach \point in {A,B,C,D}
            \fill (\point) circle (0.015);

        % Label the vertices
        \node[below left] at (A) {$A$};
        \node[below right] at (B) {$B$};
        \node[above] at (C) {$C$};
        \node[above left] at (D) {$D$};

        % Add edge weights (all same color)
        \node[below, text=blue] at ($(A)!0.5!(B)$) {$1$};
        \node[right, text=blue] at ($(B)!0.5!(C)$) {$1$};
        \node[above left, text=blue] at ($(C)!0.5!(D)$) {$w$}; % edge with weight w
        \node[above left, text=blue] at ($(D)!0.5!(A)$) {$1$};

        % Compute average y-coordinate for alignment
        \path let \p1 = (A), \p2 = (C) in
            coordinate (tableanchor) at ($(0, {(\y1 + \y2)/2})$);

        % Add cost table to the right, vertically aligned with polygon
        \node[anchor=west] at ($(B) + (.5cm, 0.5cm)$) {
            \begin{tabular}{c|c}
            Vertex & Cost \\
            \hline
            $A$ & $2$ \\
            $B$ & $w + 4$ \\
            $C$ & $w + 4$ \\
            $D$ & $w + 4$ \\
            \end{tabular}
        };

    \end{tikzpicture}
    \caption{Weights of each edge are represented in blue next to the edge, and vertex costs are denoted in the table on the right. The given budget is $B=w+4$.}
    \label{fig:greedy-cheapest}
\end{figure}\\\\
Similarly, if we wanted to greedily choose the vertex which maximizes the ratio of marginal gain in total weight to vertex cost (i.e., provides the best ``bang-for-your-buck''), we would choose vertex $A$. It guards weight of $2$ and costs $2$, a ratio of $1$. Any other vertex can guard at most $w+3$, while costing $w+4$, and $\frac{w+3}{w+4}<1$ for all $w$. Again, choosing $A$ prevents adding any more guards, achieving a total guarded weight of $2$, while an optimal strategy could guard $w+3$, and $\frac{2}{w+3}$ can be arbitrarily small for large $w$. \\\\
\cite{khuller} studies a related problem where the goal is to cover as much total weight as possible using sets whose combined cost does not exceed a given budget. They show that the greedy strategy of repeatedly selecting the set that maximizes the ratio of marginal gain in weight to cost does not guarantee any approximation ratio on its own. However, they prove that if you consider both the collection of sets selected by this greedy method and the single set that covers the most weight on its own, and return whichever of the two covers more weight, then the result is a $\frac{1}{2}(1-1/e)$-approximation. They further refine this strategy to obtain a full $(1-1/e)$-approximation.\\\\
This approach offers a direction for tackling \BMVVG{}. Specifically, we could try applying the same strategy: run the greedy algorithm that selects vertices by maximizing the ratio of marginal gain in boundary weight to cost, and also identify the single vertex that guards the most weight on its own. Then, return whichever of the two guard sets achieves greater coverage. A natural question is whether this method yields a provable approximation guarantee for \BMVVG{}, perhaps even matching the $\frac{1}{2}(1-1/e)$ bound shown in the set cover context.